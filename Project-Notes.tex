\documentclass[12pt]{article}
\usepackage{graphicx} 
\usepackage{amsmath}
\usepackage{braket}
\setlength{\parindent}{0pt}
\usepackage{diffcoeff}

\usepackage{geometry}
\geometry{
    a4paper,
    left=2cm,
    right=2cm,
    top=2cm,
    bottom=2cm,
}

\title{QM Project Notes}
\author{Nishanth S S}
\date{August 2024}

\begin{document}

\maketitle

\section{Basics of Quantum Mechanics}

\paragraph{Schrodinger's Equation:}

In quantum mechanics, we look at the particle's wave function $\psi(x,t)$, which we get by solving the Schrodinger's equation. 
\[  i \hbar \frac{\partial \psi}{\partial t} = - \frac{\hbar^2}{2m} \frac{\partial^2 \psi }{\partial x^2} + V \psi\]
\\
The Schrodinger's equation plays a role analogous to Newton's second law. Given suitable initial conditions 
(typically, $\psi(x,0)$), the Schrodinger's equation can determine $\psi(x,t)$ for all future times. 
\\
\paragraph{Probability Density and Normalization:}

The square of mod of the wavefunction gives us the probability of finding the particle at point x, at time t.
\\
\\
$ \int_a^b |\psi(x,t)|^2 dx = \int_a^b \rho (x) dx = P_{ab} = $Probability of finding particle between a and b, at time t
\\
\\
Here, $\rho(x) = |\psi(x,t)|^2 $ is the probability density of the given wave function. It follows that, the integral 
of $\psi^2$ over all x must be equal to 1. 
\[ \implies \int_{-\infty}^{\infty} |\psi(x,t)|^2 dx = 1\]
\\
We pick a multiplicative factor (complex constant) to the wave function that satisfies the above condition. This process 
is called normalizing the wave function.
\\
\\
In some cases, the solutions to the Schrodinger's equation, the integral is infinite. In these cases, no multiplicative 
factor is going to normalize the wave function. Such non-normalizable solutions cannot represent particles, and must be 
rejected. The normalization is preserved as time passes.  
\paragraph{Braket Notation:}
In quantum mechanics, we use a powerful notation introduced by Dirac. which concisely expresses our equations. 
\\
\\
Let's consider an abstract vector \textbf{a} in it's respective basis. We write it's components in that basis as a 
column matrix. Using Dirac's notation, we represent this column matrix as $\ket{\textbf{a}}$.
\\
\\
The adjoint of the column matrix \textbf{a} is given by \textbf{a^$\dagger$}. Using Dirac's notation, we represent 
this adjoint matrix as $\bra{\textbf{a}}$.
\\
\\
The scalar product between a bra and a ket is just the matrix multiplication of the two matrices. This is given by the
 notation $\braket{a|b}$.
\\
\\
The scalar product between two basis is equal to the Kronecker delta if they are orthonormal to each other. 
This is given by, $\braket{i|j} = \delta_{ij}$.
\\
\\
We now define an operator (here $\mathcal{O}$) as an entity which when acting on a ket $\ket{a}$ coverts it into a ket $\ket{b}$. 
\[ \implies \mathcal{O} \ket{a} = \ket{b}\]
\\
The adjoint of the operator $\mathcal{O}$, which we denote by $\mathcal{O^\dagger}$ changes the bra $\bra{a}$ into a bra $\bra{b}$. 
\[ \implies \bra{a} \mathcal{O^\dagger} = \bra{b}\]
\\
An operator is said to be Hermitian when it is self adjoint i.e., 
\[ \mathcal{O} = \mathcal{O^\dagger}\]
\\
The expectation value is the average of measurements on an ensemble of identically-prepared systems, not the average 
of repeated measurements on one and the same system. For a particle in state $\psi$, the expectation value of x is,
\[ \braket{x} = \int_{-\infty}^{\infty} x |\psi(x,t)|^2 dx \]
\\
To calculate the expectation value of any quantity $Q(x,p)$ where $x$ and $p = -i \hbar \frac{\partial}{\partial x}$ are the
 position and momentum operators respectively, we insert the operator between $\psi^*$ and $\psi$ and integrate:
\[ \braket{Q(x,p)} = \int \psi^* [Q(x,-i\hbar\frac{\partial}{\partial x} ] \psi dx \]


\section{Perturbation Theory and Variational Principle}

\paragraph{Perturbation Theory:}

Perturbation theory is a systematic procedure for obtaining approximate solutions to the perturbed problem, by building 
on the known exact solutions to the unperturbed case.
\\
\\
Consider the solution to the one-dimensional infinite square well (using time-independent Schrodinger's equation) which is, 
\[ H^0 \psi_n^0 = E_n^0 \psi_n^0\]
\\
Now, we perturb the potential slightly and try to find the new eigenfunctions and eigenvalues by solving the equation 
\[ H \psi_n = E_n \psi_n\]
\\
We write the new Hamiltonian of the perturbed system as $ H = H^0 + \lambda H'$ where, H' is the perturbation.
\\
We then write $\psi_n$ and $E_n$ as power series in $\lambda$. We get,
\[ \psi_n = \psi_n^0 + \lambda\psi_n^1 + \lambda^2 \psi_n^2 + ...\]
\[ E_n = E+n^0 + \lambda E_n^1 + \lambda^2 E_n^2 + ...\]
\\
Substituting the above two equaions in the perturbed time-independent Schrodinger's equation, and collecting powers of 
 $\lambda$, we get the first order approximation as

\[ H^0 \psi_n^1 + H' \psi_n^0 = E_n^0 \psi_n^1 + E_n^1 \psi^0\]
\\
Takng inner product of the above equation with $\psi_n^0$, we get
\[ \bra{\psi_n^0} H^0 \ket{\psi_n^1} + \bra{\psi_n^0} H' \ket{\psi_n^0} = E_n^0 \braket{\psi_n^0|\psi_n^1} +
 E_n^1 \braket{\psi_n^0|\psi_n^0}\]
\\
We know that, $H^0$ is Hermitian. So,
\[ \braket{\psi_n^0|H^0 \psi_n^1} = \braket{\psi_n^0 H^0| \psi_n^1} = \braket{E_n^0 \psi_n^0|\psi)n^1} = 
E_n^0 \braket{\psi_n^0|\psi_n^1}\]
\\
We know that, $\braket{\psi_n^0|\psi_n^1} = 0$ and $\braket{\psi_n^0|\psi_n^0} = 1$.
\\

\begin{equation}
\implies E_n^1 = \braket{\psi_n^0|H'|\psi_n^0}
\end{equation}
\\
The above equation is the result of first-order perturbation theory. It says that the first-order correction to 
the energy is the expectation value of the perturbation, in the unperturbed state. 
\\
\paragraph{Variational Principle:}

Given a normalized wave function $\ket{\psi}$ that satisfies the appropriate boundary condition (usually that
 requires the wave function to vanish at infinity), then the expectation value of the Hamiltonian is the upper bound to
  the to the exact ground state energy. That is if, 
\[ \braket{\psi | \psi} = 1\]
\[ \implies \bra{\psi}H\ket{\psi} \geq E_0 \]
\\
The equality holds true when $\psi$ is the ground state function i.e., $\ket{\psi} = \ket{\psi_0}$.
\\
\paragraph{Proof:}

Since the (unknown) eigenfunctions of H form a complete set, we can express as a linear combination of them.
\[ \psi = \sum_n c_n \psi_n , with \hspace{0.1cm} H \psi_n = E_n \psi_n \]
\\
Since $\psi$ is normalized, 
\[ 1 = \braket{\psi | \psi} = \braket{\sum_m c_m \psi_m | \sum_n c_n \psi_n}  = \sum_m \sum_n c_m^* c_n 
\braket{\psi_m | \psi_n} = \sum_n |c_n|^2 \]
\\
(assuming the eigenfunctions are orthonormalized i.e., $\braket{\psi_m | \psi_n} = \delta_{mn}$
\[ \implies \braket{H} = \braket{\sum_m c_m \psi_m | H \sum_n c_n \psi_n} = \sum_m \sum_n c_m^* E_n c_n 
\braket{\psi_m | \psi_n} = \sum_n E_n |c_n|^2 \]
\\
The ground state energy is defined as the smallest eigenvalue, $E_{gs} \leq E_n $
\[ \implies \braket{H} \geq E_{gs} \sum_n |c_n|^2 = E_{gs}\]


\section{Hohenberg-Kohn Theorems}

\\
\\
\paragraph{Hohenberg-Kohn Theorem I:}
The density of a non-degenerate ground state uniquely determines the external potential.

\paragraph{Proof of Theorem I:} 

Let's assume that two potentials $V_1$ and $V_2$ that differ by more than a constant give rise to the same 
electron density for a non-degenerate ground state.  
\\
\\
Let $H_1$ be $\hat{T} + \hat{V_{ee}} + \hat{V_1}$ and $\psi_1$ be ground state solution for $H_1$.
Then, $E_1 = \bra{\psi_1} H_1 \ket{\psi_1}$
\\
\\
Let $H_2$ be $\hat{T} + \hat{V_{ee}} + \hat{V_2}$ and $\psi_2$ be ground state solution for $H_2$.
Then, $E_2 = \bra{\psi_2} H_2 \ket{\psi_2}$
\\
\\
Using variational principle, we have 
\\
\begin{align*}
\bra{\psi_2}H_1 \ket{\psi_2} > E_1  \\
\\
\bra{\psi_2} H_2 + (\hat{V_1} - \hat{V_2}) \ket{\psi_2} > E_1 \\
\\
\bra{\psi_2}H_1 \ket{\psi_2} + \bra{\psi_2} (\hat{V_1} - \hat{V_2}) \ket{\psi_2} > E_1 \\
\end{align*}

\begin{equation}
    \bra{\psi_2} (\hat{V_1} - \hat{V_2}) \ket{\psi_2} > E_1 - E_2
\end{equation}

\\
Similarly, 
\begin{align*}
    \bra{\psi_1} (\hat{V_2} - \hat{V_1}) \ket{\psi_1} > E_2 - E_1\\
\end{align*}

\begin{equation}
    -\bra{\psi_1} (\hat{V_2} - \hat{V_1}) \ket{\psi_1} > E_1 - E_2
\end{equation}
\\
From (1) and (2), we have

\begin{align*}
    \bra{\psi_2} (\hat{V_1} - \hat{V_2}) \ket{\psi_2} > -\bra{\psi_1} (\hat{V_2} - \hat{V_1}) \ket{\psi_1} \\
    \\
    \int |\psi_2|^2 (V_1 - V_2) dr > - \int |\psi_1|^2 (V_2 - V_1) dr\\
    \\
    \int n_2 (r) \big[V_1 (r) - V_2 (r) \big] dr > \int n_1 (r) \big[V_1 (r) - V_2 (r) \big] dr
\end{align*}
\\
Our assumption was that the two potentials give rise to the same electron density. 
\[ \implies n_1 (r) = n_2 (r) = n(r)\]
\\
\[ \implies \int n(r) \big[V_1 (r) - V_2 (r) \big] dr > \int n(r) \big[V_1 (r) - V_2 (r) \big] dr\]
\\
This cannot be true. Hence, by contradiction theorem I is proved. 
\\
\\
\paragraph{Hohenberg-Kohn Theorem II:}
For a non-degenerate ground state, the energy is a functional of the density and is minimal at the ground state density. 

\paragraph{Proof of Theorem II:} 

Let $H_o = \hat{T} + \hat{V_{ee}} + \hat{V_{ext}}$ and $\psi_o$ be the ground state solution of $H_o$.
\\
\[ \implies \bra{\psi_o} H_o \ket{\psi_o} = E_o \]

\[\bra{\psi} H_o \ket{\psi} \geq E_o \]

\[ \bra{\psi} \hat{T} \ket{\psi} + \bra{\psi} \hat{V_{ee}} \ket{\psi} + \bra{\psi} \hat{V_{ext}} \ket{\psi} \geq E_o \]

\[ T[n] + V_{ee} [n] + V_{ext} [n] \geq E_o \]
\\
If $n = n_o$ = ground state density, then
\\
\[ T[n_o] + V_{ee} [n_o] + V_{ext} [n_o] = E_o \]

\[ \implies E[n] \geq E[n_o] \]

\section{Kohn-Sham Theory}

The Kohn-Sham theory is widely used in quantum mechanics to study the electronic structure of the many-body 
systems such as atoms and molecules. 

\\
\\
Instead of solving the many electron problem Schrodinger equation, which is computationally exhaustive due to 
the electron-electron repulsion, Kohn and Sham suggested introducing a system of non-interacting electrons that
 produces the same ground state density as the interacting electron system. 
\\
\\
They introduced a set of single particle Schrodinger like equations called the Kohn Sham equations tha describes 
the system of non-interacting electrons. The equation is given below:

\[ (\frac{-\hbar^2}{2m}\nabla^2 + V_{eff}(r))\phi_i(r) = \epsilon_i \phi_i(r)\]
\\
The above equation in atomic units will be,
\[ (\frac{-1}{2}\nabla^2 + V_{eff}(r))\phi_i(r) = \epsilon_i \phi_i(r)\]
\\
The effective potential approximates the effects of all electron-electron interactions. Thus, in the Kohn-Sham framework, 
the electrons feel the collective influence of this effective potential instead of directly interacting with each other. 
The effective potential here includes the effects of the external potential, Hartree potential and exchange correlation potential.   
\[ \implies V_{eff}(r) = V_{ext}(r) + V_H(r) + V_{xc}(r)\] ;
\[ V_H(r) = \int \frac{\rho(r')}{|r-r'|}d^3r'\]
\[ V_{xc}(r) = \frac{\delta E_{xc}[\rho]}{\delta \rho(r)}\]
\\
The solution to the Kohn Sham equations are known as Kohn Sham orbitals. These orbitals are mathematical 
constructs (fictitious) and are used in obtaining the electron density. 
\\
\\
The Kohn Sham equations are solved iteratively in a self-consistent field approach. We start with an initial 
guess for the electron density using which we calculate the effective potential. We then solve the single particle
equation to obtain the orbitals using which we compute the new density. We the compare the new density with the previous 
density and check if the difference is beyond a particular tolerance value. If not, we update the $V_{eff}$ with 
the new density and repeat the process until the difference in densities is below the tolerance.

\section{Time-Dependent Density Functional Theory}

\paragraph{Question -}
Apply perturbation theory to $\hat{\gamma}^{ks}$ where, $\hat{\gamma}^{ks} = \gamma^{oks}(t) + 
\lambda \gamma^{1}(t)$ given $\gamma^{1} = X_{ia} \ket{i} \bra{a} + Y_{ai}\ket{a} \bra{i}$ and 
$\hat{H} = H^{oks} + \lambda H^1$.

\paragraph{Solution:}
Using perturbation theory, $i \frac{\partial}{\partial t} \gamma^{ks}(t) = [H , \gamma^{ks}(t)]$

\[ i \frac{\partial}{\partial t} (\gamma^o + \lambda \gamma^1) = [ H^o + \lambda H^1 , \gamma^o + \lambda \gamma^1] \]

\[ i \frac{\partial}{\partial t} \gamma^o + i \lambda \frac{\partial}{\partial t}\gamma^1 = [H^o , 
\gamma^o] + \lambda [H^o , \gamma^1] + \lambda [H^1 , \gamma^o] + \lambda^2 [H^1 , \gamma^1] \]
\\
Equating $\lambda$ terms, we get 

\[ i \frac{\partial}{\partial t} \gamma^1 = [H^o , \gamma^1] + [H^1 , \gamma^o]\]

Separating the time-dependent part out of $\gamma^1$, we get
\[ LHS = \frac{\partial}{\partial t} \gamma^1 e^{-i \omega t} = i \gamma^1 (-i \omega) e^{-i \omega t} = 
\gamma^1 \omega e^{-i \omega t} \]  
Similarly, we get a $e^{-i\omega t}$ term on the RHS.
\\
\[ \implies \omega_\alpha \gamma^1 = [\hat{H^o} , \gamma^1] + [\hat{H^1} , \gamma^o]  
\]
We have,
\[ \gamma^1 = \sum_{ia} (X_{ia} \ket{i}\bra{a} + Y_{ai} \ket{a}\bra{i})  \]
\\
\[ \gamma^o = \sum_{m} \ket{m}\bra{m}\]
\\
\[ \hat{V} = \sum_{pq} v_{pq} \ket{p}\bra{q} \]
\\
\begin{align*}
    H_{pq}^1 &= \frac{\partial F_{pq}}{\partial \gamma_{rs}}.\gamma_{rs}^1 \\[12pt]
    &= \sum_{rs}v_{pqrs}.\gamma_{rs}^1 \\[12pt]
    &=\sum_{rsjb} v_{pqrs} . ( \bra{r} X_{jb} \ket{j}\bra{b} + Y_{bj} \ket{b}\bra{j} \ket{s}) \\[12pt]
    &=\sum_{rsjb} v_{pqrs} ( X_{jb} \delta_{rj} \delta_{bs} + Y_{bj} \delta_{rb} \delta_{js}) \\[12pt]
    &=\sum_{jb} (v_{pqjb} X_{jb} + v_{pqbj} Y_{bj}) \\[12pt]
    &\hat{H}^1 = \sum_{pq} H_{pq}^1 \ket{p}\bra{q} + \hat{V}\\[12pt]
    &=\sum_{pqjb} (v_{pqjb} X_{jb} \ket{p}\bra{q} + v_{pqbj} \ket{p}\bra{q} ) + \hat{V}
\end{align*}

Consider, 
\begin{align*}
    [\hat{H}^1 , \gamma^o] &= \hat{H}^1 \gamma^o - \gamma^o \hat{H}^1 \\[10pt]
    &= \sum_{jbpqm} \Big( (v_{pqjb} X_{jb} \ket{p}\braket{q|m}\bra{m} + v_{pqbj} Y_{bj} \ket{p}\braket{q|m}\bra{m}) \boldsymbol{-} \\
    &\quad (v_{pqjb} X_{jb} \ket{m}\braket{m|p}\bra{q} + v_{pqbj} Y_{bj} \ket{m}\braket{m|p}\bra{q}) \Big) \\[10pt]
    &= \sum_{jbpqm} \Big( v_{pqjb} X_{jb} \ket{p}\delta_{qm}\bra{m} + v_{pqbj} Y_{bj} \ket{p}\delta_{qm}\bra{m} 
     - v_{pqjb} X_{jb} \ket{m}\delta_{mp}\bra{q} - v_{pqbj} Y_{bj} \ket{m}\delta_{mp}\bra{q} \Big) \\[10pt]
    &= \sum_{jbpm} \Big( v_{pmjb} X_{jb} \ket{p} \bra{m} + v_{pmbj} Y_{bj} \ket{p}\bra{m} \Big) 
     - \sum_{jbqm} \Big( v_{mqjb} X_{jb} \ket{m}\bra{q} + v_{mqbj} Y_{bj} \ket{m}\bra{q} \Big) \\[10pt]
    &= \sum_{jbpm} \Big( v_{pmjb} X_{jb} \ket{p} \bra{m} + v_{pmbj} Y_{bj} \ket{p}\bra{m} 
     - v_{mpjb} X_{jb} \ket{m}\bra{p} - v_{mpbj} Y_{bj} \ket{m}\bra{p} \Big)
\end{align*}

Considering the potential separately, we get 

\begin{align*}
    [\hat{V},\gamma^o] &= \hat{V} \gamma^o - \gamma^o \hat{V} \\[10pt]
    &= \sum_{pqm} \Big( v_{pq }\ket{p}\braket{q|m}\bra{m} - \ket{m}\braket{m|p}\bra{q} v_{pq} \Big)\\[10pt]
    &= \sum_{pqm} \Big( v_{pq }\ket{p}\delta_{qm}\bra{m} - \ket{m}\delta_{mp}\bra{q} v_{pq} \Big) \\[10pt]
    &= \sum_{pm} v_{pm} \ket{p}\bra{m} - \sum_{qm} v_{mq} \ket{m}\bra{q}
\end{align*}

\begin{align*}
    [\hat{H}^o , \gamma^1] &= \hat{H}^o \gamma^1 - \gamma^1 \hat{H}^o \\[10pt]
    &= \sum_{kc} \Big( H^o ( X_{kc} \ket{k}\bra{c} + Y_{ck} \ket{c}\bra{k}) - (X_{kc} \ket{k}\bra{c} + 
    Y_{ck} \ket{c}\bra{k})H^o \Big) \\[10pt]
    &=  \sum_{kc} \Big(  X_{kc} H^o\ket{k}\bra{c} + Y_{ck} H^o\ket{c}\bra{k} - X_{kc} \ket{k}\bra{c}H^o - Y_{ck} 
    \ket{c}\bra{k}H^o \Big)\\[10pt]
    &=  \sum_{kc} \Big(  X_{kc} \epsilon_k \ket{k}\bra{c} + Y_{ck} \epsilon_c \ket{c}\bra{k} - X_{kc} \ket{k}\bra{c} 
    \epsilon_c - Y_{ck} \ket{c}\bra{k} \epsilon_k \Big)\\[10pt]
    &= \sum_{kc} \Big( (\epsilon_k - \epsilon_c) X_{kc} \ket{k} \bra{c} - (\epsilon_k - \epsilon_c) Y_{ck} \ket{c}\bra{k} \Big)
\end{align*}

Combining all the above equations, we get
\begin{align*}
\omega_\alpha \Big( \sum_{ia} (X_{ia} \ket{i}\bra{a} + Y_{ai} \ket{a}\bra{i}) \Big) &=  \sum_{jbpm} 
\Big( v_{pmjb} X_{jb} \ket{p} \bra{m} + v_{pmbj} Y_{bj} \ket{p}\bra{m} \\[12pt]
     &- v_{mpjb} X_{jb} \ket{m}\bra{p} - v_{mpbj} Y_{bj} \ket{m}\bra{p} \Big) + \\[12pt] 
     &\sum_{kc} \Big( (\epsilon_k - \epsilon_c) X_{kc} \ket{k} \bra{c} - (\epsilon_k - \epsilon_c) 
     Y_{ck} \ket{c}\bra{k} \Big) \\[12pt]
     &+ \sum_{pm} v_{pm} \ket{p}\bra{m} - \sum_{qm} v_{mq} \ket{m}\bra{q}
\end{align*}

Projecting with $\braket{l|\hat{O}|d}$, we get
\[ \omega_\alpha X_{ld} = - \sum_{jb} \Big( v_{ldjb} X_{jb} + v_{ldbj} Y_{bj} \Big) 
    + (\epsilon_l - \epsilon_d) X_{ld}  + v_{ld}^{ext}\]
\\
Projecting with $\braket{d|\hat{O}|l}$, we get
\[ \omega_\alpha Y_{dl} = \sum_{jb} \Big( v_{dljb} X_{jb} + v_{dlbj} Y_{bj} \Big) 
    - (\epsilon_l - \epsilon_d) Y_{dl} + \tilde{v}_{dl}^{ext} \]

Thus, we have
\[ \sum_{jb} (\omega_{dl} \delta_{db} \delta_{lj} + v_{ldjb}) X_{jb} + \sum_{jb} v_{ldbj} Y_{bj} + 
\omega_\alpha \sum_{jb} \delta_{db} \delta_{lj} X_{jb} =  v_{ld}^{ext}\]

\[ \sum_{jb} (\omega_{dl} \delta_{db} \delta_{lj} + v_{dlbj}) Y_{bj} + \sum_{jb} v_{dljb} X_{jb} - 
\omega_\alpha \sum_{jb} \delta_{db} \delta_{lj} Y_{bj} = -\tilde{v}_{dl}^{ext} \]
where, $\omega_{dl} = \epsilon_d - \epsilon_l$
\\
We can write the above equation in the matrix form : 
\[
\left[
\begin{bmatrix}
    A & B \\
    B^* & A^*
\end{bmatrix}
- \omega_\alpha
\begin{bmatrix}
    1 & 0 \\
    0 & -1
\end{bmatrix}
\right]
\begin{bmatrix}
    X \\
    Y
\end{bmatrix}
=
\begin{bmatrix}
    v_{ext} \\
    \tilde{v}_{ext}
\end{bmatrix}
\]
where, $A,B,B*$ and $A*$ are matrices.


\end{document}
